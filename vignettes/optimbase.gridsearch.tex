\inputencoding{utf8}
\HeaderA{optimbase.gridsearch}{Grid evaluation of a constrained or unconstrained cost function}{optimbase.gridsearch}
\keyword{method}{optimbase.gridsearch}
%
\begin{Description}\relax
Evaluate a constrained or unconstrained cost function on a grid of points 
around a given initial point estimate.
\end{Description}
%
\begin{Usage}
\begin{verbatim}
  optimbase.gridsearch(fun = NULL, x0 = NULL, xmin = NULL, 
                       xmax = NULL, npts = 3, alpha = 10)
\end{verbatim}
\end{Usage}
%
\begin{Arguments}
\begin{ldescription}
\item[\code{fun}] A constrained or unconstrained cost function defined as described 
in the vignette (\code{vignette('optimbase',package='optimbase')}).
\item[\code{x0}] The initial point estimate, provided as a numeric vector.
\item[\code{xmin}] Optional: a vector of lower bounds.
\item[\code{xmax}] Optional: a vector of upper bounds.
\item[\code{npts}] A integer scalar greater than 2, indicating the number of 
evaluation points will be used on each dimension to build the search grid.
\item[\code{alpha}] A vector of numbers greater than 1, which give the factor(s) used 
to calculate the evaluation range of each dimension of the search grid (see 
Details). If \code{alpha} length is lower than that of \code{x0}, elements 
of \code{alpha} are recycled. If its length is higher than that of 
\code{x0}, \code{alpha} is truncated.
\end{ldescription}
\end{Arguments}
%
\begin{Details}\relax
\code{optimbase.gridsearch} evaluates the cost function at each point 
of a grid of \code{npts\textasciicircum{}length(x0)} points. If lower (\code{xmin}) and upper 
(\code{xmax}) bounds are provided, the range of evaluation points is limited 
by those bounds and \code{alpha} is not used. Otherwise, the range of 
evaluation points is defined as \code{[x0/alpha,x0*alpha]}.

\code{optimbase.gridsearch} also determines if the cost function is
feasible at each evaluation point by calling \code{optimbase.isfeasible}.
\end{Details}
%
\begin{Value}
Return a data.frame with the coordinates of the evaluation point, the value of
the cost function and its feasibility. The data.frame is ordered by 
feasibility and increasing value of the cost function.
\end{Value}
%
\begin{Author}\relax
Sebastien Bihorel (\email{sb.pmlab@gmail.com})
\end{Author}
%
\begin{SeeAlso}\relax
\code{\LinkA{optimbase.isfeasible}{optimbase.isfeasible}}
\end{SeeAlso}
%
\begin{Examples}
\begin{ExampleCode}

# Problem: find x and y that maximize 3.6*x - 0.4*x^2 + 1.6*y - 0.2*y^2 and
#          satisfy the constrains:
#            2*x - y <= 10
#            x >= 0
#            y >= 0
#

gridfun <- function(x=NULL,index=NULL,fmsfundata=NULL,...){

  f <- c()
  c <- c()
  if (index == 2 | index == 6)
    f <- -(3.6*x[1] - 0.4*x[1]*x[1] + 1.6*x[2] - 0.2*x[2]*x[2])
  if (index == 5 | index == 6)
    c <- c(10 - 2*x[1] - x[2],
           x[1],
           x[2])
  varargout <- list(f = f, g = c(), c = c, gc = c(), index = index)

  return(varargout)
}


x0 <- c(0.35,0.3)
npts <- 6
alpha <- 10

res <- optimbase.gridsearch(fun=gridfun,x0=x0,xmin=NULL,xmax=NULL,
                     npts=npts,alpha=alpha)

# 3.5 and 3 is the actual solution of the optimization problem
print(res)

\end{ExampleCode}
\end{Examples}
