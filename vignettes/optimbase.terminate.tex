\inputencoding{utf8}
\HeaderA{optimbase.terminate}{Evaluation of Termation Status}{optimbase.terminate}
\keyword{method}{optimbase.terminate}
%
\begin{Description}\relax
This function determines whether the optimization must continue or terminate.
If the \code{verbosetermination} element of the optimization object is
enabled, messages are printed detailing the termination intermediate steps.
The \code{optimbase.terminate} function takes into account the number of
iterations, the number of evaluations of the cost function, the tolerance on x
and the tolerance on f. See the section "Termination" in
\code{vignette('optimbase',package='optimbase')} for more details.
\end{Description}
%
\begin{Usage}
\begin{verbatim}
  optimbase.terminate(this = NULL, previousfopt = NULL, currentfopt = NULL,
                      previousxopt = NULL, currentxopt = NULL)
\end{verbatim}
\end{Usage}
%
\begin{Arguments}
\begin{ldescription}
\item[\code{this}] An optimization object.
\item[\code{previousfopt}] The previous value of the objective function.
\item[\code{currentfopt}] The current value of the objective function.
\item[\code{previousxopt}] The previous value of the parameter estimate matrix.
\item[\code{currentxopt}] The current value of the parameter estimate matrix.
\end{ldescription}
\end{Arguments}
%
\begin{Value}
Return a list with the following elements: \begin{description}

\item[this] The updated optimization object.
\item[terminate] TRUE if the algorithm terminates, FALSE if the algorithm
must continue.
\item[status] The termination status could be 'maxiter', 'maxfuneval',
'tolf' or 'tolx' if \code{terminate} is set to TRUE, 'continue'
otherwise.

\end{description}

\end{Value}
%
\begin{Author}\relax
Author of Scilab optimbase module: Michael Baudin (INRIA - Digiteo)

Author of R adaptation: Sebastien Bihorel (\email{sb.pmlab@gmail.com})
\end{Author}
