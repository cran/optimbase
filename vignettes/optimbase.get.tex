\inputencoding{utf8}
\HeaderA{Get}{Get the value for the given key}{Get}
\aliasA{optimbase.cget}{Get}{optimbase.cget}
\aliasA{optimbase.get}{Get}{optimbase.get}
\aliasA{optimbase.histget}{Get}{optimbase.histget}
\keyword{method}{Get}
%
\begin{Description}\relax
Get the value for the given key in an optimization object. 
\end{Description}
%
\begin{Usage}
\begin{verbatim}
  optimbase.get(this = NULL, key = NULL)
  optimbase.cget(this = NULL, key = NULL)
  optimbase.histget(this = NULL, iter = NULL, key = NULL)
\end{verbatim}
\end{Usage}
%
\begin{Arguments}
\begin{ldescription}
\item[\code{this}] An optimization object.
\item[\code{key}] The name of the key to quiery. The list of available keys for
query with \code{optimbase.get} is: '-funevals', '-iterations', '-xopt',
'-fopt', '-historyxopt', '-historyfopt', '-fx0', '-status', and
'-logstartup'.

The list of available keys for query with \code{optimbase.cget} is:
'-verbose', '-verbosetermination', '-function', '-method', '-x0',
'-maxfunevals', '-maxiter', '-tolfunabsolute', '-tolfunrelative',
'-tolxabsolute', '-tolxrelative', '-tolxmethod', '-tolfunmethod',
'-outputcommand', '-outputcommandarg', '-numberofvariables',
'-storehistory', '-costfargument', '-boundsmin', '-boundsmax',
'-nbineqconst', '-logfile', and '-withderivatives'.

The list of available keys for query with \code{optimbase.histget} is:
'-historyxopt' and '-historyfopt'.

\item[\code{iter}] The iteration at which the data is stored.
\end{ldescription}
\end{Arguments}
%
\begin{Details}\relax
\code{optimbase.get} extracts the value of elements which are not available
directly to the user interface, but are computed internally, while
\code{optimbase.cget} extracts the value of elements which are available to
the user interface. While \code{optimbase.get} extracts the entire content of
\code{historyxopt} and \code{historyfopt}, \code{optimbase.histget} only
extracts the content of the history at the iteration \code{iter}.
\end{Details}
%
\begin{Value}
Return the value of the list element \code{key}, or an error message if
\code{key} does not exist.  
\end{Value}
%
\begin{Author}\relax
Author of Scilab optimbase module: Michael Baudin (INRIA - Digiteo)

Author of R adaptation: Sebastien Bihorel (\email{sb.pmlab@gmail.com})
\end{Author}
%
\begin{SeeAlso}\relax
\code{\LinkA{optimbase}{optimbase}},
\code{\LinkA{optimbase.configure}{optimbase.configure}}
\end{SeeAlso}
