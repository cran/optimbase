\inputencoding{utf8}
\HeaderA{optimbase.get}{Get the value for the given element}{optimbase.get}
\aliasA{optimbase.histget}{optimbase.get}{optimbase.histget}
\keyword{method}{optimbase.get}
%
\begin{Description}\relax
Get the value for the given element in an optimization object. 
\end{Description}
%
\begin{Usage}
\begin{verbatim}
  optimbase.get(this = NULL, key = NULL)
  optimbase.histget(this = NULL, iter = NULL, key = NULL)
\end{verbatim}
\end{Usage}
%
\begin{Arguments}
\begin{ldescription}
\item[\code{this}] An optimization object.
\item[\code{key}] The name of the key to quiery. The list of available keys for
query with \code{optimbase.get} is: 'verbose', 'x0', 'fx0', 'xopt', 'fopt',
'tolfunabsolute', 'tolfunrelative', 'tolfunmethod', 'tolxabsolute',
'tolxrelative', 'tolxmethod', 'maxfunevals', 'maxiter', 'iterations',
'function', 'status', 'historyfopt', 'historyxopt', 'verbosetermination',
'outputcommand', 'outputcommandarg', 'numberofvariables', 'storehistory',
'costfargument', 'boundsmin', 'boundsmax', 'nbineqconst', 'logfile',
'logfilehandle', 'logstartup', and'withderivatives'.

The list of available keys for query with \code{optimbase.histget} is:
'historyxopt' and 'historyfopt'.

\item[\code{iter}] The iteration at which the data is stored.
\end{ldescription}
\end{Arguments}
%
\begin{Details}\relax
While \code{optimbase.get} extracts the entire content of the object element,
including \code{historyxopt} and \code{historyfopt}, \code{optimbase.histget}
only extracts the content of the history at the iteration \code{iter}.
\end{Details}
%
\begin{Value}
Return the value of the list element \code{key}, or an error message if
\code{key} does not exist.
\end{Value}
%
\begin{Author}\relax
Author of Scilab optimbase module: Michael Baudin (INRIA - Digiteo)

Author of R adaptation: Sebastien Bihorel (\email{sb.pmlab@gmail.com})
\end{Author}
%
\begin{SeeAlso}\relax
\code{\LinkA{optimbase}{optimbase}},
\code{\LinkA{optimbase.set}{optimbase.set}}
\end{SeeAlso}
