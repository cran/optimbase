\inputencoding{utf8}
\HeaderA{Bound and constraint checks}{Point Estimate Comparison with Bounds and Constraints}{Bound and constraint checks}
\aliasA{optimbase.isinbounds}{Bound and constraint checks}{optimbase.isinbounds}
\aliasA{optimbase.isinnonlincons}{Bound and constraint checks}{optimbase.isinnonlincons}
\keyword{method}{Bound and constraint checks}
%
\begin{Description}\relax
\code{optimbase.isinbounds} checks that given parameter estimates are within
the defined minimum and maximum boundaries, while
\code{optimbase.isinnonlincons} checks that the given point estimate satisfies
the defined nonlinear constraints.
\end{Description}
%
\begin{Usage}
\begin{verbatim}
  optimbase.isinbounds(this = NULL, x = NULL)
  optimbase.isinnonlincons(this=NULL,x=NULL)
\end{verbatim}
\end{Usage}
%
\begin{Arguments}
\begin{ldescription}
\item[\code{this}] An optimization object.
\item[\code{x}] A column vector of parameter estimates.
\end{ldescription}
\end{Arguments}
%
\begin{Value}
Both functions return a list with the following elements: \begin{description}

\item[this] The optimization object.
\item[isfeasible] TRUE if the parameter estimates satisfy the constraints,
FALSE otherwise.

\end{description}

\end{Value}
%
\begin{Author}\relax
Author of Scilab optimbase module: Michael Baudin (INRIA - Digiteo)

Author of R adaptation: Sebastien Bihorel (\email{sb.pmlab@gmail.com})
\end{Author}
