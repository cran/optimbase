\inputencoding{utf8}
\HeaderA{optimbase.checkcostfun}{Check Cost Function}{optimbase.checkcostfun}
\keyword{method}{optimbase.checkcostfun}
%
\begin{Description}\relax
This function checks that the cost function is correctly specified in the
optimization object, including that the elements of \code{this} used by the
cost function are consistent.
\end{Description}
%
\begin{Usage}
\begin{verbatim}
  optimbase.checkcostfun(this = NULL)
\end{verbatim}
\end{Usage}
%
\begin{Arguments}
\begin{ldescription}
\item[\code{this}] An optimization object
\end{ldescription}
\end{Arguments}
%
\begin{Details}\relax
Depending on the definition of nonlinear constraints (\code{nbineqconst}
element > 0) and the use of derivatives (\code{withderivatives} element set to
TRUE), this function makes several cost function calls with different
\code{index} value (see \code{vignette('optimbase',package='optimbase')} for
more details about \code{index}). If  at least one call fails, the function
stops the search algorithm.

Following every successful cost function call, \code{optimbase.checkcostfun}
calls \code{optimbase.checkshape} to check the dimensions of the matrix
returned by the cost function against some expectations.
\end{Details}
%
\begin{Value}
Return the optimization object or an error message if one check is not
successful.
\end{Value}
%
\begin{Author}\relax
Author of Scilab optimbase module: Michael Baudin (INRIA - Digiteo)

Author of R adaptation: Sebastien Bihorel (\email{sb.pmlab@gmail.com})
\end{Author}
%
\begin{SeeAlso}\relax
\code{\LinkA{optimbase.checkshape}{optimbase.checkshape}}
\end{SeeAlso}
