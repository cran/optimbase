\inputencoding{utf8}
\HeaderA{optimbase.isfeasible}{Check Point Estimate}{optimbase.isfeasible}
\keyword{method}{optimbase.isfeasible}
%
\begin{Description}\relax
This function checks that the point estimate is consistent with the bounds
and the non linear inequality constraints. It is usually called by  
\code{optimbase.checkx0} to check initial guesses.
\end{Description}
%
\begin{Usage}
\begin{verbatim}
  optimbase.isfeasible(this = NULL, x = NULL)
\end{verbatim}
\end{Usage}
%
\begin{Arguments}
\begin{ldescription}
\item[\code{this}] An optimization object.
\item[\code{x}] The point estimate, i.e. a column vector of numerical values.
\end{ldescription}
\end{Arguments}
%
\begin{Details}\relax
Returns 1 if the given point satisfies bounds constraints and inequality
constraints.

Returns 0 if the given point is not in the bounds.

Returns -1 if the given point does not satisfies inequality constraints.
\end{Details}
%
\begin{Value}
Return a list with the following elements: \begin{description}

\item[this] The optimization object.
\item[isfeasible] The feasibility flag, either -1, 0 or 1.

\end{description}

\end{Value}
%
\begin{Author}\relax
Author of Scilab optimbase module: Michael Baudin (INRIA - Digiteo)

Author of R adaptation: Sebastien Bihorel (\email{sb.pmlab@gmail.com})
\end{Author}
%
\begin{SeeAlso}\relax
\code{\LinkA{optimbase.checkx0}{optimbase.checkx0}}
\end{SeeAlso}
