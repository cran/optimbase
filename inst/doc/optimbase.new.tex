\inputencoding{utf8}
\HeaderA{optimbase.new}{Initialize optimbase Object}{optimbase.new}
\keyword{method}{optimbase.new}
%
\begin{Description}\relax
This function creates a new object. 
\end{Description}
%
\begin{Usage}
\begin{verbatim}
  optimbase.new()
\end{verbatim}
\end{Usage}
%
\begin{Value}
Return a new optimization object, i.e. a list with a 'type' attribute set to
'T\_OPTIMIZATION' and containing the following elements:\begin{description}

\item[verbose] The verbose option, controlling the amount of messages.
Set to 0.
\item[x0] The initial guess. Set to NULL.
\item[fx0] The value of the function for the initial guess. Set to
NULL.
\item[xopt] The optimum parameter. Set to 0.
\item[fopt] The optimum function value. Set to 0.
\item[tolfunabsolute] The absolute tolerance on function value. Default
is 0.
\item[tolfunrelative] The relative tolerance on function value. Default
is .Machine\$double.eps.
\item[tolfunmethod] Logical flag for the tolerance on function value in
the termination criteria. This criteria is suitable for functions which
minimum is associated with a function value equal to 0. Set to FALSE.
\item[tolxabsolute] The absolute tolerance on x. Set to 0.
\item[tolxrelative] The relative tolerance on x. Set to
.Machine\$double.eps.
\item[tolxmethod] Possible values: FALSE, TRUE. Set to TRUE.
\item[funevals] The number of function evaluations. Set to 0.
\item[maxfunevals] The maximum number of function evaluations. Set to
100.
\item[maxiter] The maximum number of iterations. Set to 100.
\item[iterations] The number of iterations. Set to 0.
\item[fun] The cost function. Set to ''.
\item[status] The status of the optimization. Set to ''.
\item[historyfopt] The vector to store the history for fopt. The values of
the cost function will be stored at each iteration in a new element, so
the length of \code{historyfopt} at the end of the optimization should be
the number of iterations. Set to NULL.
\item[historyxopt] The list to store the history for xopt. The vectors of
estimates will be stored on separated levels of the list, so the length of
\code{historyfopt} at the end of the optimization should be the number of
iterations. Set to NULL.
\item[verbosetermination] The verbose option for termination criteria.
Set to 0.
\item[outputcommand] The command called back for output. Set to ''.
\item[outputcommandarg] The outputcommand argument is initialized as a
string. If the user configure this element, it is expected that a matrix
of values or a list is passed so that the argument is appended to the name
of the function. Set to ''.
\item[numberofvariables] The number of variables to optimize. Set to
0.
\item[storehistory] The flag which enables/disables the storing of the
history. Set to FALSE.
\item[costfargument] The costf argument is initialized as a string. If
the user configure this element, it is expected that a matrix of values
or a list is passed so that the argument is appended to the name of the
function. Set to ''.
\item[boundsmin] Minimum bounds for the parameters. Set to NULL.
\item[boundsmax] Maximum bounds for the parameters. Set to NULL.
\item[nbineqconst] The number of nonlinear inequality constraints. Default
is 0.
\item[logfile] The name of the log file. Set to ''.
\item[logfilehandle] The handle for the log file. Set to 0.
\item[logstartup] Set to TRUE when the logging is started up. Set to
FALSE.
\item[withderivatives] Set to TRUE when the method uses derivatives.
Set to FALSE.

\end{description}

\end{Value}
%
\begin{Author}\relax
Author of Scilab optimbase module: Michael Baudin (INRIA - Digiteo)

Author of R adaptation: Sebastien Bihorel (\email{sb.pmlab@gmail.com})
\end{Author}
