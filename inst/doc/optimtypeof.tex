\inputencoding{utf8}
\HeaderA{optimtypeof}{Object Type or Class}{optimtypeof}
\keyword{method}{optimtypeof}
%
\begin{Description}\relax
This functions extracts the \code{type} attribute of lists commonly created
by \code{optimbase.new}.
\end{Description}
%
\begin{Usage}
\begin{verbatim}
  optimtypeof(object = NULL)
\end{verbatim}
\end{Usage}
%
\begin{Arguments}
\begin{ldescription}
\item[\code{object}] Any object but usually a optimization, simplex or neldermead
list object.
\end{ldescription}
\end{Arguments}
%
\begin{Details}\relax
If \code{object} is not a list or if \code{type} is not an attribute of
\code{object}, \code{optimtypeof} returns the class of \code{object}.
\end{Details}
%
\begin{Value}
Returns a single character string which is either the content of the
\code{type} attribute of \code{object} or its class.
\end{Value}
%
\begin{Author}\relax
Sebastien Bihorel (\email{sb.pmlab@gmail.com})
\end{Author}
%
\begin{SeeAlso}\relax
\code{\LinkA{optimbase.new}{optimbase.new}}
\end{SeeAlso}
%
\begin{Examples}
\begin{ExampleCode}
  obj1 <- optimbase.new()
  optimtypeof(obj1)
  optimtypeof(obj1$optbase)
  optimtypeof(obj1$simplex0)

  obj2 <- list(1)
  attr(obj2, 'type') <- 'newtype'
  optimtypeof(obj2)
\end{ExampleCode}
\end{Examples}
