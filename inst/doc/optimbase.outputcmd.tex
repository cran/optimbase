\inputencoding{utf8}
\HeaderA{optimbase.outputcmd}{Call user-defined output function}{optimbase.outputcmd}
\keyword{method}{optimbase.outputcmd}
%
\begin{Description}\relax
Call user-defined output function.
\end{Description}
%
\begin{Usage}
\begin{verbatim}
  optimbase.outputcmd(this = NULL, state = NULL, data = NULL)
\end{verbatim}
\end{Usage}
%
\begin{Arguments}
\begin{ldescription}
\item[\code{this}] An optimization object.
\item[\code{state}] The current state of the algorithm: either 'init', 'iter', or
'done'.
\item[\code{data}] A list containing at least the following elements:
\begin{description}

\item[x] the current point estimate,
\item[fval] the value of the cost function at the current point
estimate,
\item[iteration] the current iteration index,
\item[funccount] the number of function evaluations.

\end{description}


\end{ldescription}
\end{Arguments}
%
\begin{Details}\relax
The \code{data} list argument may contain more levels than those presented
above. These additionnal levels may contain values which are specific to the
specialized algorithm, such as the simplex in a Nelder-Mead method, the
gradient of the cost function in a BFGS method, etc...
\end{Details}
%
\begin{Value}
Do not return any data, but execute the output function defined in the
\code{outputcommand} element of \code{this}.
\end{Value}
%
\begin{Author}\relax
Author of Scilab optimbase module: Michael Baudin (INRIA - Digiteo)

Author of R adaptation: Sebastien Bihorel (\email{sb.pmlab@gmail.com})
\end{Author}
