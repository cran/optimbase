\inputencoding{utf8}
\HeaderA{asserts}{Check of Variable Class}{asserts}
\aliasA{assert.typeboolean}{asserts}{assert.typeboolean}
\aliasA{assert.typefunction}{asserts}{assert.typefunction}
\aliasA{assert.typereal}{asserts}{assert.typereal}
\aliasA{assert.typestring}{asserts}{assert.typestring}
\aliasA{unknownValueForOption}{asserts}{unknownValueForOption}
\keyword{method}{asserts}
%
\begin{Description}\relax
Utility functions in \pkg{optimbase} meant to check variable class. Stop the
algorithm if the  variable is not of the expected class.\begin{description}

\item[\code{assert.typeboolean}] for logical variables
\item[\code{assert.typefunction}] for functions
\item[\code{assert.typereal}] for numeric variables
\item[\code{assert.typestring}] for character variables

\end{description}

\code{unknownValueForOption} stops the algorithm and returns an error
message, when some checks in \code{optimbase} are not successful.
\end{Description}
%
\begin{Usage}
\begin{verbatim}
  assert.typeboolean(var = NULL, varname = NULL, ivar = NULL)
  assert.typefunction(var = NULL, varname = NULL, ivar = NULL)
  assert.typereal(var = NULL, varname = NULL, ivar = NULL)
  assert.typestring(var = NULL, varname = NULL, ivar = NULL)
  unknownValueForOption(value = NULL, optionname = NULL)
\end{verbatim}
\end{Usage}
%
\begin{Arguments}
\begin{ldescription}
\item[\code{var}] The variable name.
\item[\code{varname}] The name of a variable to which \code{var} should have been
assigned to.
\item[\code{ivar}] A integer, meant to provide additional info on \code{varname} in
the error message.
\item[\code{value}] A numeric or a string.
\item[\code{optionname}] The name of a variable for which \code{value} is unknown.
\end{ldescription}
\end{Arguments}
%
\begin{Value}
Return an error message through the \code{stop} function.
\end{Value}
%
\begin{Author}\relax
Author of Scilab optimbase module: Michael Baudin (INRIA - Digiteo)

Author of R adaptation: Sebastien Bihorel (\email{sb.pmlab@gmail.com})
\end{Author}
