\inputencoding{utf8}
\HeaderA{optimbase.configure}{Optimization Object Configuration}{optimbase.configure}
\aliasA{optimbase.histset}{optimbase.configure}{optimbase.histset}
\aliasA{optimbase.set}{optimbase.configure}{optimbase.set}
\keyword{method}{optimbase.configure}
%
\begin{Description}\relax
This functions configures the current optimization object with the given
\code{value} for the given \code{key}.
\end{Description}
%
\begin{Usage}
\begin{verbatim}
  optimbase.configure(this = NULL, key = NULL, value = NULL)
  optimbase.set(this = NULL, key = NULL, value = NULL)
  optimbase.histset(this = NULL, iter = NULL, key = NULL, value = NULL)
\end{verbatim}
\end{Usage}
%
\begin{Arguments}
\begin{ldescription}
\item[\code{this}] The current optimization object.
\item[\code{key}] The key to configure. See details for the list of possible keys.
\item[\code{value}] The value to assign to the key.
\item[\code{iter}] The iteration at which the data must be stored.
\end{ldescription}
\end{Arguments}
%
\begin{Details}\relax
\code{optimbase.configure} and \code{optimbase.set} set the content of the
\code{key} element of the optimization object \code{this} to \code{value}.

The only available keys in \code{optimbase.configure} are the following:
\begin{description}

\item['-verbose'] Set to 1 to enable verbose logging.
\item['-verbosetermination'] Set to 1 to enable verbose termination
logging.
\item['-x0'] The initial guesses, as a n x 1 column vector, where n is the
number of variables.
\item['-maxfunevals'] The maximum number of function evaluations. If this
criteria is triggered during optimization, the status of the optimization
is set to 'maxfuneval' (see
\code{vignette('optimbase',package='optimbase')} for more details).
\item['-maxiter'] The maximum number of iterations. If this criteria is
triggered during optimization, the status of the optimization is set to
'maxiter' (see \code{vignette('optimbase',package='optimbase')}
for more details).
\item['-tolfunabsolute'] The absolute tolerance for the function value.
\item['-tolfunrelative'] The relative tolerance for the function value.
\item['-tolfunmethod'] The method used for the tolerance on function value
in the termination criteria. The following values are available: TRUE,
FALSE. If this criteria is triggered, the status of the optimization is
set to 'tolf'.
\item['-tolxabsolute'] The absolute tolerance on x.
\item['-tolxrelative'] The relative tolerance on x.
\item['-tolxmethod'] The method used for the tolerance on x in the
termination criteria. The following values are available: TRUE, FALSE. If
this criteria is triggered during optimization, the status of the
optimization is set to 'tolx'.
\item['-function'] The objective function, which computes the value of the
cost function and the non linear constraints, if any. See
\code{vignette('optimbase',package='optimbase')} for the details of
the communication between the optimization system and the cost function.
\item['-costfargument'] An additionnal argument, passed to the cost
function.
\item['-outputcommand'] A command which is called back for output. Details
of the communication between the optimization system and the output
command function are provided in      
\code{vignette('optimbase',package='optimbase')}.
\item['-outputcommandarg'] An additionnal argument, passed to the output
command.
\item['-numberofvariables'] The number of variables to optimize.
\item['-storehistory'] Set to TRUE to enable the history storing.
\item['-boundsmin'] The minimum bounds for the parameters.
\item['-boundsmax'] The maximum bounds for the parameters.
\item['-nbineqconst'] The number of inequality constraints.
\item['-logfile'] The name of the log file.
\item['-withderivatives'] Set to TRUE if the algorithm uses derivatives.

\end{description}


The only available keys in \code{optimbase.set} are the following:\begin{description}

\item['-iterations'] the number of iterations.
\item['-xopt'] the optimum point estimate.
\item['-fopt'] the value of the cost function at the optimum point 
estimate.
\item['-historyxopt'] a list, with nbiter element, containing the history
of x during the iterations. This list is available after optimization if
the history storing was enabled with the \code{storehistory}
element.
\item['-historyfopt'] an vector, with nbiter values, containing the history
of the function value during the iterations. This vector is available
after optimization if the history storing was enabled with the
\code{storehistory} element.
\item['-fx0'] the value of the cost function at the initial point
estimate.
\item['-status'] a string containing the status of the optimization.

\end{description}


The only available keys in \code{optimbase.histset} are '-historyxopt' and
'-historyfopt'. Contrary to \code{optimbase.set}, this function only alters
the value of \code{historyxopt} and \code{historyfopt} at the specific
iteration \code{iter}.
\end{Details}
%
\begin{Value}
An updated optimization object.
\end{Value}
%
\begin{Author}\relax
Author of Scilab optimbase module: Michael Baudin (INRIA - Digiteo)

Author of R adaptation: Sebastien Bihorel (\email{sb.pmlab@gmail.com})
\end{Author}
%
\begin{SeeAlso}\relax
\code{\LinkA{optimbase.new}{optimbase.new}}
\end{SeeAlso}
