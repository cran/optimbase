\inputencoding{utf8}
\HeaderA{optimbase.outstruct}{Create Basic Optimization Data Object}{optimbase.outstruct}
\keyword{method}{optimbase.outstruct}
%
\begin{Description}\relax
This function creates a basic optimization data object by extracting the
content of specific fields of an optimization object.
\end{Description}
%
\begin{Usage}
\begin{verbatim}
  optimbase.outstruct(this = NULL)
\end{verbatim}
\end{Usage}
%
\begin{Arguments}
\begin{ldescription}
\item[\code{this}] An optimization object.
\end{ldescription}
\end{Arguments}
%
\begin{Value}
Return a list with a 'type' attribute set to 'T\_OPTDATA' and with the
following elements: \begin{description}

\item[x] The current optimum point estimate (extracted from
\code{this\$xopt}).
\item[fval] The value of the cost function at the current optimum point
estimate (extracted from \code{this\$fopt}).
\item[iteration] The current number of iteration (extracted
from \code{this\$iterations}).
\item[funccount] The current number of function evaluations (extracted from
\code{this\$funevals}).

\end{description}

\end{Value}
%
\begin{Author}\relax
Author of Scilab optimbase module: Michael Baudin (INRIA - Digiteo)

Author of R adaptation: Sebastien Bihorel (\email{sb.pmlab@gmail.com})
\end{Author}
