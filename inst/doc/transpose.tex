\inputencoding{utf8}
\HeaderA{transpose}{Vector and Matrix Transpose}{transpose}
\keyword{method}{transpose}
%
\begin{Description}\relax
\code{transpose} is a wrapper function around the \code{t} function, which
tranposes matrices. Contrary to \code{t}, \code{transpose} processes vectors
as if they were row matrices.
\end{Description}
%
\begin{Usage}
\begin{verbatim}
  transpose(object = NULL)
\end{verbatim}
\end{Usage}
%
\begin{Arguments}
\begin{ldescription}
\item[\code{object}] A vector or a matrix.
\end{ldescription}
\end{Arguments}
%
\begin{Value}
Return a matrix which is the exact transpose of the vector or matrix \code{x}
\end{Value}
%
\begin{Author}\relax
Sebastien Bihorel (\email{sb.pmlab@gmail.com})
\end{Author}
%
\begin{SeeAlso}\relax
\code{\LinkA{t}{t}}
\end{SeeAlso}
%
\begin{Examples}
\begin{ExampleCode}
  1:6
  t(1:6)
  transpose(1:6)
  mat <- matrix(1:15,nrow=5,ncol=3)
  mat
  transpose(mat)
\end{ExampleCode}
\end{Examples}
