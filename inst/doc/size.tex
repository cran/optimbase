\inputencoding{utf8}
\HeaderA{size}{Vector, Matrix or Data.Frame Size}{size}
\keyword{method}{size}
%
\begin{Description}\relax
\code{size} is a utility function which determines the dimensions of vectors
(coerced to matrices), matrices, arrays, data.frames, and list elements.
\end{Description}
%
\begin{Usage}
\begin{verbatim}
  size(x = NULL, n = NULL)
\end{verbatim}
\end{Usage}
%
\begin{Arguments}
\begin{ldescription}
\item[\code{x}] A R object.
\item[\code{n}] A integer indicating the dimension of interest.
\end{ldescription}
\end{Arguments}
%
\begin{Details}\relax
\code{size} is a wrapper function around \code{dim}. It returns the n\textasciicircum{}th
dimension of \code{x} if \code{n} is provided. If \code{n} is not provide,
all dimensions will be determined. If \code{x} is a list, \code{n} is ignored
and the dimensions of all elements of \code{x} are recursively determined.
\end{Details}
%
\begin{Value}
Returns a vector or list of dimensions.
\end{Value}
%
\begin{Author}\relax
Sebastien Bihorel (\email{sb.pmlab@gmail.com})
\end{Author}
%
\begin{SeeAlso}\relax
\code{\LinkA{dim}{dim}}
\end{SeeAlso}
%
\begin{Examples}
\begin{ExampleCode}
  a <- 1
  b <- letters[1:6]
  c <- matrix(1:20,nrow=4,ncol=5)
  d <- array(1:40, dim=c(2,5,2,2))
  e <- data.frame(a,b)
  f <- list(a,b,c,d,e)

  size(NULL) # 0 0
  size(NA)   # 1 1
  size(a)    # 1 1
  size(b,2)  # 6
  size(c)    # 4 5
  size(d)    # 2 5 2 2
  size(e,3)  # NA
  size(f)
\end{ExampleCode}
\end{Examples}
